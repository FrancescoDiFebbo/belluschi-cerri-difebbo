
\section*{doLoadFromFile( )}

\lstset{firstnumber=457}
\begin{lstlisting}
 /**
 * Loads any currently active sessions that were previously unloaded
 * to file
 *
 * @exception ClassNotFoundException if a serialized class cannot be
 * found during the reload
 * @exception IOException if a read error occurs
 */
private void doLoadFromFile() throws ClassNotFoundException, IOException {
	if (log.isLoggable(Level.FINE)) {
		log.log(Level.FINE, "Start: Loading persisted sessions");
	}

	// Open an input stream to the specified pathname, if any
	File file = file();
	if (file == null || !file.exists() || file.length() == 0) {
		return;
	}
	if (log.isLoggable(Level.FINE)) {
		String msg = MessageFormat.format(rb.getString(LOADING_PERSISTED_SESSION), pathname);
		log.log(Level.FINE, msg);
	}
	FileInputStream fis = null;
	try {
		fis = new FileInputStream(file.getAbsolutePath());
		readSessions(fis);
		if (log.isLoggable(Level.FINE)) {
			log.log(Level.FINE, "Finish: Loading persisted sessions");
		}
	} catch (FileNotFoundException e) {
		if (log.isLoggable(Level.FINE)) {
			log.log(Level.FINE, "No persisted data file found");
		}
	} finally {
	try {
		if (fis != null) {
			fis.close();
		}
	} catch (IOException f) {
		// ignore
	}
	// Delete the persistent storage file
	deleteFile(file);
	}
}

\end{lstlisting}

\paragraph{Problems}
\begin{enumerate}
	\item @ 495 catch not managed [53]
\end{enumerate}

\paragraph{Other Problems}
There is two main problem that must be fixed in order to have an easier debug:

\begin{enumerate}
	\item the catch statement @ 495 must be managed
	\item @ 473 is better to use the log to inform the caller that the method returns without perform any operation
\end{enumerate}


\section*{readSessions(InputStream is)}



\lstset{firstnumber=509}
\begin{lstlisting}
/*
* Reads any sessions from the given input stream, and initializes the
* cache of active sessions with them.
*
* @param is the input stream from which to read the sessions
*
* @exception ClassNotFoundException if a serialized class cannot be
* found during the reload
* @exception IOException if a read error occurs
*/
public void readSessions(InputStream is)
		throws ClassNotFoundException, IOException {

	// Initialize our internal data structures
	sessions.clear();

	ObjectInputStream ois = null;
	try {
		BufferedInputStream bis = new BufferedInputStream(is);
		if (container != null) {
			ois = ((StandardContext)container).createObjectInputStream(bis);
		} else {
			ois = new ObjectInputStream(bis);
		}
	} catch (IOException ioe) {
		String msg = MessageFormat.format(rb.getString(LOADING_PERSISTED_SESSION_IO_EXCEPTION),
										      ioe);

		log.log(Level.SEVERE, msg, ioe);
		if (ois != null) {
			try {
				ois.close();
			} catch (IOException f) {
				// Ignore
			}
			ois = null;
		}
		throw ioe;
	}

	synchronized (sessions) {
		try {
			Integer count = (Integer) ois.readObject();
			int n = count.intValue();
			if (log.isLoggable(Level.FINE))
				log.log(Level.FINE, "Loading " + n + " persisted sessions");
			for (int i = 0; i < n; i++) {
				StandardSession session =
					StandardSession.deserialize(ois, this);
				session.setManager(this);
				sessions.put(session.getIdInternal(), session);
				session.activate();
			}
		} catch (ClassNotFoundException e) {
			String msg = MessageFormat.format(rb.getString(CLASS_NOT_FOUND_EXCEPTION),
												  e);
			log.log(Level.SEVERE, msg, e);
			if (ois != null) {
				try {
					ois.close();
				} catch (IOException f) {
					// Ignore
				}
				ois = null;
			}
			throw e;
		} catch (IOException e) {
			String msg = MessageFormat.format(rb.getString(LOADING_PERSISTED_SESSION_IO_EXCEPTION),
												  e);
		  log.log(Level.SEVERE, msg, e);
			if (ois != null) {
				try {
					ois.close();
				} catch (IOException f) {
					// Ignore
				}
				ois = null;
			}
			throw e;
		} finally {
			// Close the input stream
			try {
				if (ois != null) {
					ois.close();
			}
			} catch (IOException f) {
				// ignore
			}
		}
	}
}

\end{lstlisting}

\paragraph{Problems}
\begin{enumerate}
	\item @ 578 two spaces instead of four are used for indentation [8]
	\item @ 553-4 the if statement is not surronded by curly braces [11]
	\item @ 566-573, @ 579-586, @590-596 duplicated code [27]
	\item @ 541, @ 569, @ 582, @ 594 catch not managed [53]
\end{enumerate} 

\paragraph{Other Problems}

There are three main problems that must be fixed in order to avoid some bugs:

\begin{enumerate}
	\item the duplicated code  @ 566-573, @ 579-586 can be deleted because the finally statement does the same thing
	\item the catch statement @ 541, @ 569, @ 582, @ 594 must be managed
	\item there is a dead code @ 538-545; the variable ois, @ 538, is null so the rest of the code is never executed
\end{enumerate} 
