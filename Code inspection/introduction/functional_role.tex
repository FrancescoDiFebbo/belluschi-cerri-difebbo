The class inspected is \textbf{StandardManager}.\\It belongs to the package \textit{org.apache.catalina.session}\\
The class inheritance is the following:
\begin{verbatim}
java.lang.Object
    org.apache.catalina.session.ManagerBase
        org.apache.catalina.session.StandardManager
\end{verbatim}

Implemented interfaces are:
\begin{verbatim}
PropertyChangeListener, EventListener, Lifecycle, Manager
\end{verbatim}
\lstset{firstnumber=79}
\begin{lstlisting}
/**
* Standard implementation of the <b>Manager</b> interface that provides
* simple session persistence across restarts of this component (such as
* when the entire server is shut down and restarted, or when a particular
* web application is reloaded.
* <p>
* <b>IMPLEMENTATION NOTE</b>:  Correct behavior of session storing and
* reloading depends upon external calls to the <code>start()</code> and
* <code>stop()</code> methods of this class at the correct times.
*
* @author Craig R. McClanahan
* @author Jean-Francois Arcand
* @version $Revision: 1.14.6.2 $ $Date: 2008/04/17 18:37:20 $
*/
\end{lstlisting}
\paragraph{Description} \textit{StandardManager} inherits from \textit{ManagerBase}, a minimal implementation of the \textit{Manager} interface that supports no session persistence or distributable capabilities. On the contrary, \textit{StandardManager} manages the persistence of sessions when occurs start or stop events on the component. \textit{Lifecycle} interface is responsible for component lifecycle methods. Components may implement this interface in order to provide a consistent mechanism to start and stop the components. \textit{PropertyChangeListener} interface is responsible for \textit{PropertyChange} events handling whenever a bean changes a "bound" property.\\\\\\



\title{{\large \textbf{setContainer(Container container)}}}
\lstset{firstnumber=245}
\begin{lstlisting}
/**
* Set the Container with which this Manager has been associated.  If
* it is a Context (the usual case), listen for changes to the session
* timeout property.
*
* @param container The associated Container
*/
@Override
public void setContainer(Container container)
\end{lstlisting}

\paragraph{Description} This method overrides \textit{ManagersBase}'s \textit{setContainer} method. In particular, it de-registers from any pre-existing, \textit{Context Container} through the \textit{removePropertyChangeListener} method. In any case, the method then proceeds calling the superclass \textit{setContainer} method. Finally, if the \textit{Container} argument that was passed is a non-null \textit{Context}, the method registers with it through the \textit{addPropertyChangeListener} method.\\\\\\


\title{{\large \textbf{load( )}}}
\lstset{firstnumber=426}
\begin{lstlisting}
/**
* Loads any currently active sessions that were previously unloaded
* to the appropriate persistence mechanism, if any.  If persistence is not
* supported, this method returns without doing anything.
*
* @exception ClassNotFoundException if a serialized class cannot be
* found during the reload
* @exception IOException if a read error occurs
*/
public void load() throws ClassNotFoundException, IOException
\end{lstlisting}

\paragraph{Description} This method requires persistence support (verifying it through the \textit{SecurityUtil.isPackageProtectionEnabled()} method), otherwise simply calls \textit{doLoadFromFile}. The method looks for and loads all previously unloaded active sessions: it tries the \textit{AccessController.doPrivileged(PrivilegedExceptionAction$\langle T \rangle$)} method, catching exceptions \textit{ClassNotFoundException}, \textit{IOException} and reporting all other unexpected ones. \\\\\\


\title{{\large \textbf{doLoadFromFile( )}}}
\lstset{firstnumber=457}
\begin{lstlisting}
/**
* Loads any currently active sessions that were previously unloaded
* to file
*
* @exception ClassNotFoundException if a serialized class cannot be
* found during the reload
* @exception IOException if a read error occurs
*/
private void doLoadFromFile() throws ClassNotFoundException, IOException 
\end{lstlisting}

\paragraph{Description} This method loads persisted session from an existing file. If the file returned by the \textit{file()} method doesn't exist, is empty or is null the method returns and no operations are performed. Otherwise a \textit{FileInputSession fis} is open with parameter the \textit{absolutePath} of the file. If no exceptions are caught the method invokes the \textit{readSession(fis)} otherwise a error message is shown by the logger. In any case the method closes the \textit{FileInputStream} and if there are no other exceptions the method deletes the file.\\\\\\


\title{{\large \textbf{readSessions(InputStream is)}}}
\lstset{firstnumber=509}
\begin{lstlisting}
/*
* Reads any sessions from the given input stream, and initializes the
* cache of active sessions with them.
*
* @param is the input stream from which to read the sessions
*
* @exception ClassNotFoundException if a serialized class cannot be
* found during the reload
* @exception IOException if a read error occurs
*/
public void readSessions(InputStream is)
throws ClassNotFoundException, IOException
\end{lstlisting}

\paragraph{Description} This method reads any session from the parameter \textit{is} (\textit{InputStream} object). First of all the method initializes the class variable sessions. Then, the method, creates a \textit{ObjectInputStream ois} and try to initialize it with a \textit{BufferedInputStream bis} from the \textit{InputStream is} (the parameter of the method). If an \textit{IOException} is caught the method shows an error through the logger and then try to close the \textit{ObjectInputStream ois} created. Then the class variable sessions is synchronized and the method try to load the persisted sessions and create a object session from the class \textit{StardardSession} and put it in the sessions class variable and activates it. In any case the method closes the \textit{ObjectInputStream} and terminates.\\\\\\


\title{{\large \textbf{unload(boolean doExpire, boolean isShutDown)}}}
\lstset{firstnumber=626}
\begin{lstlisting}
/**
* Save any currently active sessions in the appropriate persistence
* mechanism, if any.  If persistence is not supported, this method
* returns without doing anything.
*
* @doExpire true if the unloaded sessions are to be expired, false
* otherwise
* @param isShutdown true if this manager is being stopped as part of a
* domain shutdown (as opposed to an undeployment), and false otherwise
*
* @exception IOException if an input/output error occurs
*/
protected void unload(boolean doExpire, boolean isShutdown) throws IOException
\end{lstlisting}
\paragraph{Description} This method checks if package protection mechanism is enabled through \textit{SecurityUtil.isPackageProtectionEnabled()} method. If so \textit{doUnload(boolean, boolean)} method is performed with privileges enabled through \textit{AccessController} \textit{.doPrivileged(PrivilegedExceptionAction$\langle T \rangle$)} method. If the action's method throws an exception, it will propagate through the latter method. Otherwise \textit{doUnload(boolean, boolean)} method is simply executed. If an exception occurs, it will be rethrown without handling.\\\\\\

\title{{\large \textbf{doUnload(boolean doExpire, boolean isShutdown)}}}
\lstset{firstnumber=657}
\begin{lstlisting}
/**
* Saves any currently active sessions to file.
*
* @doExpire true if the unloaded sessions are to be expired, false
* otherwise
*
* @exception IOException if an input/output error occurs
*/
private void doUnload(boolean doExpire, boolean isShutdown) throws IOException
\end{lstlisting}
\paragraph{Description} This method checks \textit{isShutDown} parameter. If it's true the session is saved. A new file (through \textit{file()} method) and its output stream are opened. On this output stream are written all active sessions through \textit{writeSession(OutputStream, boolean)} method. If \textit{isShutDown} is false or file is not valid or an exception occurs sessions will not be written.

