\subsection{Registration}
\paragraph{Purpose}
Visitors can register to myTaxiService through the web or mobile application. They can register either as a passenger or as a taxi driver.\\
In both cases, this operation requires the visitor to fill a registration form with personal data and accept myTaxiService terms and conditions, including personal data policies, according to local law. In case of registration as a taxi driver, the system requires the visitor more info, including proof of the possession of a valid taxi driver license.\\
If any of the previous requirements are not met or any input is invalid, the registration fails and the system asks the visitor to repeat the process. Otherwise, a verification email is sent to the provided email address: from that email the visitor can confirm his new account and successfully end the registration process.

\paragraph{Scenarios}
\begin{enumerate}
	\item Alex is a student. He has heard about myTaxiService and, finding it an easy way to travel, wants to subscribe to it.\\
	Therefore, he access to the homepage of the web application, clicks "Register", then chooses "Passenger". He fulfills the form, accepts the terms and conditions, and clicks "Confirm". However, the system cannot verify Alex's info because the confirmation password does not match with the first one. It therefore asks Alex to write it again. This time Alex fills the form correctly, then clicks "Confirm". The system verifies his info, then sends Alex a verification email to the submitted email address. Alex checks his mailbox, opens the new mail and clicks on the link inside it, redirecting him back to the web application of myTaxiService. The system informs him that the registration has successfully ended.
	He can now log in as a passenger user.
	
	\item Bob is a taxi driver. His company recommends him to subscribe to myTaxiService, in order to make his work easier and improve the taxi service.\\
	Therefore, he downloads and installs the mobile app of myTaxiService on his mobile phone, then opens it. He taps "Register", then chooses "Taxi Driver". He inputs all the required data, including his driver license ID, accepts the terms and conditions and confirms. The system verifies the submitted info and sends Bob a confirmation email. Bob checks his mailbox, opens the new mail and taps on the link inside. The system informs him that the registration has successfully ended. He can now log in as a taxi driver user.
\end{enumerate}

\paragraph{Diagrams}

\begin{center}
	\includegraphics[height=\textwidth]{diagrams/registration}
\end{center}
\begin{center}
	\includegraphics[width=\textwidth]{diagrams/registration_state}
\end{center}

\paragraph{Functional Requirements}
\begin{itemize}
	
	\item Visitors can register either as passengers or as taxi drivers.	
	\item Visitors can abort the registration process at any time.	
	\item The link in the confirmation email must be clicked within 1 day, otherwise the registration is deleted along with the visitor's info.	
	\item Registration forms contain the following info (fields):
	\begin{itemize}
		\item email address
		\item username
		\item password
		\item password confirmation
		\item name
		\item surname
		\item (*) address
		\item (*) telephone number
		\item (**) taxi license ID
		\item (**) taxi plate number
		\item (**) taxi code
	\end{itemize}
	All fields must be contain valid inputs.\\
	Fields marked with (*) are not mandatory.\\
	Fields marked with (**) are only for taxi driver registrations.	
	\item email address and username cannot be the same as ones from other myTaxiService users.	
	\item password must contain at least 8 characters.	
	\item password and password confirmation must match.	
\end{itemize}

\subsection{Login}

\paragraph{Purpose}
Visitors on myTaxiService website or mobile application may access to an existing registered user account providing its corresponding username (or email address) and password. In case the submitted info do not match with any existing account info, the system notifies the visitor that the username/email address doesn't exist, or that it exists, but the submitted password is wrong.\\
In case a user forgets his/her password, the system allows him/her to retrieve it, automatically creating a new password, setting it as the user's one and sending it to the provided email address.

\paragraph{Scenarios}
\begin{enumerate}
	\item Carl is a passenger user. He opens myTaxiService website, but can't remember his password to access the service. Therefore, he clicks on "Forgotten password?". The system asks him for the email address or username he provided at registration. He writes it down and clicks "Confirm". The system verifies the existence of the submitted email address, then creates a new password and sends it in an email to the submitted email address.
	
	\item Daisy is a passenger user, familiar with the myTaxiService website. She wants to use the mobile app, too. Knowing that she can enter either username or email and password, she fills both fields and clicks on "Log in". The system verifies her info: the operation ends successfully, and she gains access to the passenger user homepage.
\end{enumerate}

\paragraph{Diagrams}
\begin{center}
	\includegraphics[height=\textwidth]{diagrams/login}
\end{center}
\begin{center}
	\includegraphics[width=\textwidth]{diagrams/login_state}
\end{center}
\paragraph{Functional Requirements}

\subsection{Standard ride request}
\paragraph{Purpose}
Passenger users can request a taxi both through the web or through the mobile application, giving only simple data about the number of passengers and sharing preferences (in case of shared ride, see also ?.?.?, "Shared Ride Request").\\
In any case, the system will then care about keeping the user informed about all details of his request, i.e. status of the request, estimated time of arrival (ETA) of the incoming taxi, in addition to its taxi code.
\paragraph{Scenarios}
\begin{enumerate}
	\item Elsa wanted to take the bus, but the heavy snow that fell in the last three days caused a lot of traffic problems. Fortunately for her, the taxi service is still functioning, so she opens myTaxiService on her mobile phone, logs in, and chooses "Request".
	She uses the GPS info to fill the "Origin" field, leaves the "Share" checkbox blank, then "Confirm". In a matter of minutes Frank, a taxi driver in her zone, accepts her request: Elsa is informed that she has to wait approximately 6 minutes for her taxi, encoded 288, to arrive. In the meanwhile, the system give her updates about the taxi position. At the expected time Frank arrives, picks Elsa up and carries her to desired destination.
\end{enumerate}
\paragraph{Diagrams}
\begin{center}
	\includegraphics[height=\textwidth]{diagrams/standard_request}
\end{center}
\paragraph{Functional Requirements}
\begin{itemize}
	
	\item The system allows taxi ride requests if and only if the passenger accepts to give info about his/her location, either through GPS or directly writing down a valid location.
	
	\item The system allows taxi ride requests if and only if the passenger can be located in some definite position of some definite taxi zone.
	
	\item The system uses default values for the number of passengers and sharing preferences of a ride (1 person, no sharing), unless the passenger does specify them.
	
	\item The system uses a FIFO policy to manage forwarding of pending ride requests.
	
	\item The system uses a FIFO policy to manage the order of taxi drivers in queues to send notifications to.
	
	\item The system forwards a ride request to the first taxi driver in the considered zone queue if and only if he/she has a sufficient number of free seats available in his/her vehicle.
	
	\item The system keeps the passenger(s) notified about the status of the ride request he/she sent.
	
	\item Once a ride request has been accepted by some taxi driver, the system changes the request status from "Pending" to "Accepted".
	
	\item Once a ride request has been accepted by some taxi driver, the system calculates the ETA of the incoming taxi based on the distance between the taxi and the passenger(s), and the current traffic.
	
	\item Once a ride request has been accepted by some taxi driver, the system notifies the passenger(s) about the ETA of the incoming taxi.
	
	\item Once a ride request has been accepted by some taxi driver, the system keeps the passenger(s) notified about the current location of the incoming taxi, showing its position on a map.
	
	\item Once a ride request has been accepted by some taxi driver, the system prevents the passenger(s) to make a new ride request until the taxi driver changes the status of the ride to "Completed".
\end{itemize}

\subsection{Reserved ride request}

\paragraph{Purpose}
Passenger users can request to reserve a taxi for some definite future ride. The operation can be done both through the web or through the mobile application, and requires information about the location and exact date and time of the meeting point, the destination, the number of passengers and the sharing preferences (in case of shared ride, see also ?.?.?, "Shared Ride Request").\\
In any case, the system will then care about sending a taxi to the given location at the given date and time. Reservation requests must occur at least two hours before the ride meeting time.

\paragraph{Scenarios}
\begin{enumerate}
	\item George has an important meeting tomorrow morning, but his car suddenly broke. He decides he will take a taxi. Therefore, he opens the homepage of myTaxiService web application on his laptop, logs in as a passenger user, then clicks "Reserve". He selects "use maps" for both position fields, and pinpoints his home and the location of the meeting as "Origin" and "Destination", respectively. He selects "7.15" as the meeting time, leaves the "Share" checkbox blank, then clicks "Confirm".\\
	The next day, at 7.05, a reserved ride requests is received by Harry, the first taxi driver in the queue of the taxi zone where George's meeting point is located. Harry decides to refuse the request, though. The request is then forwarded to Isabelle, which was the second taxi driver in queue at the time of Harry's refusal. She accepts George's request, and at the given time arrives at his house. She picks him up and brings him to the meeting.
\end{enumerate}

\paragraph{Diagrams}

\begin{center}
	\includegraphics[height=\textwidth]{diagrams/reserved_request}
\end{center}

\paragraph{Functional Requirements}

\subsection{Shared ride request}

\paragraph{Purpose}

\paragraph{Scenarios}

\paragraph{Diagrams}

\paragraph{Functional Requirements}

\subsection{Request notification and response}

\paragraph{Purpose}

\paragraph{Scenarios}

\paragraph{Diagrams}

\paragraph{Functional Requirements}

\subsection{Availability settings}

\paragraph{Purpose}
Taxi drivers are able to notify the system about their status through the mobile application at any moment, as long as they're logged in. In particular, the status can be either "Ready", "Busy" or "Offline".\\
Whenever a taxi driver logs in, the system automatically sets his/her status from "Offline" to "Ready" and put him/her on the bottom of its current taxi zone queue, based on GPS info.\\
When he/she accepts a taxi ride, the status is automatically updated to "Busy": the system then removes him/her from the queue, preventing the arrival of other ride requests.\\
Similarly, when the ride is over, the taxi driver has to notify the system that the ride has ended: the system automatically changes the status back to "Ready" and puts him/her back on the bottom of the current taxi zone queue, thus waiting for a new ride request.\\
Finally, when the taxi driver finishes his workshift, he may inform the system, or simply log off. In both cases, his/her status automatically switches to "Offline".

\paragraph{Scenarios}
Albert is a taxi driver subscribed to myTaxiService. He logs in through his mobile phone and his status changes from "Offline" to "Ready". The system receives info from the GPS and puts Albert on the bottom of the taxi zone he's currently in. After a while, his phone notifies him about a new ride request: he decides to accept it and his status changes to "Busy". He's no longer in the taxi queue. Albert goes to the start location, picks up Barbara and takes her to her destination. When they arrive, Albert informs the system that he has concluded the ride: his status changes to "Ready". The system puts him on the bottom of his current taxi zone queue. Later on, he receives another ride requests, but this time he decides to refuse it: its status remains unchanged as "Ready", but he loses all his positions in the queue.
A few hours later, Albert finishes his worktime and logs off. The system sets his status to "Offline" and removes him from any queue.

\paragraph{Diagrams}
\begin{center}
	\includegraphics[width=\textwidth]{diagrams/availability}
\end{center}
\paragraph{Functional Requirements}
\begin{itemize}
	
	\item The system uses a FIFO policy to manage taxi zone queues.
	
	\item The system uses info provided by the GPS to locate taxis and decide their respective queues.
	
	\item The system automatically inserts taxi drivers in queues when their status changes to "Ready".
	
	\item The system automatically removes taxi drivers from queues when their status changes to "Busy" or "Offline".
	
	\item The status automatically changes to "Busy" when the taxi driver accepts a ride request.
	
	\item The status automatically changes to "Ready" when the taxi driver notifies the end of a ride.
	
	\item When status is "Ready", the application notifies about ride requests.
	
	\item When status is "Ready", the application enables the taxi driver to accept/refuse requests.
	
	\item When status is "Busy", the application prevents ride requests notifications.
	
	\item When status is "Busy", the application enables the taxi driver to notify the end of the current ride.
	
\end{itemize}

\subsection{Account Settings}

\paragraph{Purpose}
The system allows registered users to view and modify their profiles at any moment, as long as they're logged in. Usernames cannot be modified, while modified email addresses, taxi license IDs and taxi codes must not match with the ones of other users, otherwise the system denies the modification request. In case of modified email address, the system sends a confirmation email to the new address. Modification will succesfully ends when the user clicks the link in the sent email. 

\paragraph{Scenarios}
Alex uses to periodically change his account password, in order to increase protection. To do so, every 3 months, he opens myTaxiService on his mobile phone, chooses "Profile", then "Modify". He selects the password field, writes down a new one, then writes it again in the "Confirm password" field. Finally, he clicks "Confirm": the system informs him that his account password has succesfully been updated.

\paragraph{Diagrams}
\begin{center}
	\includegraphics[width=\textwidth]{diagrams/account_settings}
\end{center}
\paragraph{Functional Requirements}
