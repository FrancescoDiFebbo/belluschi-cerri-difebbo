This chapter describes the software system, the relationships between software components and the relationship to actors with the system. Each component is described by a specification and an interface design. The specification is a description of its purpose, its functionality, its attributes (including dependency on other components) and the constraints under which it must operate. It also describes resources, that is, any elements used by the component which are external to the design such as physical devices and software services. The interface design is the list of the services that it provides to clients. These services are methods (procedures and functions), each carefully documented.\\Each component in turn may provide its services by having an internal architectural design with its own set of subordinate components. These components may be called sub-components. The decomposition of a higher-level component into subordinate component must be explicit. The algorithm that shows how each method of the larger component is performed by these components must be explicit. Any data stored in an component must be explicitly described.
