Detailed descriptions of scale drivers are provided by COCOMO manual, mentioned before.

\begin{itemize}
	
	\item\textbf{Precedentedness (PREC):} our team lacks of a previous experience using these development methodologies, thus the value will be \textit{very low}.
	
	\item\textbf{Development flexibility (FLEX):} this project requires general specifications without going deep in details, thus the value will be \textit{nominal}.
	
	\item\textbf{Architecture/Risk resolution (RESL):} our architecture provides security mechanisms and risk preventions, thus the value will be \textit{high}.
	
	\item\textbf{Team cohesion (TEAM):} our team is composed by members who had a previous working experience together, thus the value will be \textit{very high}.
	
	\item\textbf{Process Maturity (PMAT):} reflects the process maturity of the organization. Our CMM level corresponds to \textit{high} value. 
	
\end{itemize}

\vspace{10mm}

\begin{center}
	\begin{tabular}{| l | c | c |}
		\hline
		\textbf{Scale driver} & \textbf{Factor} & \textbf{Value} \\ \hline
		PREC & $Very\ low$ & 6.20\\ \hline
		FLEX & $Nominal$ & 3.04\\ \hline
		RESL & $High$ & 2.83\\ \hline
		TEAM & $Very\ high$ & 2.19\\ \hline
		PMAT & $High$ & 3.12\\ \hline
		\multicolumn{2}{|l|}{\textbf{Total points}} & 17.38\\ \hline
	\end{tabular}
\end{center}