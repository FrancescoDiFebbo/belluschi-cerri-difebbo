The Functional Point approach is a technique that allows to evaluate the effort
needed for the design and implementation of a project. We have used this technique
to evaluate the application dimension basing on the functionalities of the
application itself. The functionalities list has been obtained from the RASD document
and for each one of them the realization complexity has been evaluated.
The functionalities has been groped in:
\begin{itemize}
\item Internal Logic File: it represents a set of homogeneous data handled by the system
\item External Interface File: it represents a set of homogeneous data used
by the application but handled by external application
\item External Input: elementary operation that allows input of data in the system
\item External Output: elementary operation that creates a bit stream towards
the outside of the application
\item External Inquiry: elementary operation that involves input and output operations 
\end{itemize}

\newpage

The following table outline the number of Functional Point based on functionality
and relative complexity:

\begin{table}[h]
	\centering
	\begin{tabular}{| l | l | l | l |}
		\hline
		\multirow{2}{*}{\textbf{Function Type}} & \multicolumn{3}{c|}{\textbf{Complexity}} \\
		\cline{2-4}
		& Simple & Medium & Complex  \\
		\hline
		Internal Logic File & 7 & 10 & 15 \\ \hline
		External Interface File & 5 & 7 & 10 \\ \hline
		External Input & 3 & 4 & 6 \\ \hline
		External Output & 4 & 5 & 7 \\ \hline
		External Inquiry & 3 & 4 & 6 \\ \hline	
		
	\end{tabular}
\end{table}

